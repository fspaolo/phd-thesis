
% Chapter 1 - Introduction 
% ------------------------
% Jun 6, 2015


\chapter{Introduction and background}

\section*{Ice shelves, ice-sheet and sea-level rise}

\lettrine[lines=2]{T}{he Antarctic Ice Sheet loses} most of its mass from its fringing ice shelves\footnote{
The ice shelves loose their mass through either iceberg calving or basal melting
(see Figure \ref{ch1fig1}).},
the marginal floating extensions of inland glaciers and ice streams. More than 80\% of
Antarctica's ice drains through the ice shelves, where most of the mass
lost from the ice sheet is transferred to the ocean. The Antarctic Ice Sheet
contains ice equivalent to $\sim$58 m of global sea-level rise
\parencite{Fretwell2013}, and has been losing mass at an average rate of
$\sim$71 Gt year$^{-1}$ (gigatonnes per year) between 1992 and 2011
\parencite{Shepherd2012}\footnote{This is equivalent to a contribution of
(slightly over) 6\% to the total sea-level rise of $\sim$3.2 mm year$^{-1}$
(CITE).}. A moderate loss of 2\% of the Antarctic Ice Sheet is sufficient to rise
the global sea level by $\sim$1 m. More important, however, is the accelerated
state of this ice-loss rate, as has been observed during the past decade
\parencite{Shepherd2012, Sutterley2014, Velicogna2009, Chen2009, Harig2015}, 
which raises significant concern on future ice-sheet behaviour and its
contribution to global sea level. 

Located at the boundary between the ice sheet, atmosphere and ocean,
Antarctica's ice shelves are potentially vulnerable to changes in both
atmospheric and oceanic conditions (Figure \ref{ch1fig1}), making them sensitive indicators of
large-scale climate change. For more than two decades, rapid
changes have been occurring in the extent and thickness of many Antarctic ice
shelves, particularly along the Antarctic Peninsula and the Amundsen Sea
sector of West Antarctica \parencite{Cook2010, Pritchard2012, Shepherd2010,
Wingham2009, Zwally2005, Fricker2012}(Figure \ref{ch1fig2}[some figure
showing ice-shelf loss?]). It is, therefore, through the ice shelves that variations in
oceanic and atmospheric states force changes in the grounded ice sheet.

In the past decade, considerable progress has been made in understanding the
fundamental role that the ice shelves play in restraining the grounded ice sheet
flow [Schoof; Goldberg; Hilmar; etc.]. Ice shelves exert a back-stress on
grounded tributary-glaciers and ice streams resulting from drag forces
at the ice-bedrock interface. This resistive stress, known as \emph{buttressing effect},
holds back the ice flow from the ice-sheet interior to the ocean.
With loss of back-stress due to ice-shelf shrinkage or breakup, ice discharge
increases scaling nonlinearly with grounding line (GL)\footnote{The ground line
is the dynamic boundary between the grounded ice sheet and the floating ice shelves;
where glaciers detach from the bedrock and start afloat.} thickness to
compensate for the reduction in \emph{buttressing} [7, 21]. This nonlinear dynamics,
flow $\propto$ thickness$^{n \geqslant 3}_\text{GL}$, becomes particularly important in
regions where (a) the ice sheet is
grounded below sea level (marine ice sheet) and (b) the bed deepens inwards
(retrograde bed slope). Such configuration gives rise to a condition of unstable
equilibrium known as \emph{marine ice-sheet instability} (Figure \ref{ch1fig3}),
first proposed in the 70s by [see Hilmar for the right citations].

There have been recent rapid advances in identifying the dynamical processes by which the
ocean can control ice-sheet mass loss and associated sea-level rise. Studies have shown that
increased ocean-forced basal melting of ice shelves, accelerating flow in adjacent grounded ice,
is responsible for the majority of current Antarctic ice sheet loss \parencite{Rignot2008,
Pritchard2009}. Dramatic grounding line retreat as a consequence
of intensified basal melting has induced extensive land ice thinning \parencite{Wingham2009,
Pritchard2009, Rignot2014}, and faster flow rates have followed ice shelf collapses
due to the reduction in \emph{buttressing} \parencite{Rignot2004, Rignot2005, Scambos2004}.
Indeed, 87\% of all Antarctic Peninsula (AP) tidewater glaciers are known to be retreating
\parencite{Cook2005}, and dynamic thinning (increase strain rates due to flow acceleration)
has occurred around the AP \parencite{Rignot2008, Pritchard2009}.
These observations have been interpreted as evidence that
ocean forcing can lead to rapid changes in ice sheet dynamic flow and its subsequent
contribution to sea-level rise. Despite this progress, our understanding of these processes
is still too rudimentary to allow prediction of ice sheet change under projected
future climate states.

In essence, because the grounded ice sheet changes in response to perturbations in the ice
shelves, understanding variations in the state of the ice shelves is key to
identify relationships between observed ice loss and large-scale climate variability.
There are two complementary ways forward: to develop our understanding of the actual mass
loss processes, so they can be better represented in models; and to empirically relate observed
ice-sheet change to ocean and atmospheric variability. This dissertation focuses on the second
approach.

\section*{Satellite altimetry and change detection}

Continuous observations of ice shelves over long time periods are required to determine stability,
monitor change, and identify general relationships between observed changes and ocean
variability. Given the vast size of Antarctica, its remote location and challenging field conditions,
space-based techniques are the only practical way to monitor the ice sheet.
Much of our current understanding of how ice-shelf processes couple ice-sheet changes to climate variability comes from analyses of trends in surface elevation change, $\partial h / \partial t$, derived from satellite radar and laser altimeter data \parencite{Zwally2005, Shepherd2010, Pritchard2012, Fricker2012}.
In particular, satellite radar altimeters\footnote{Radar antennas mounted on satellites flying at
$\sim$780 km of altitude.}
have provided the longest set of continuous observations over the Antarctic and Greenland ice sheets, and have revolutionized the way we study these ice masses. Maps of ice-shelf height change at high spatial resolution have also been developed using measurements from a satellite laser altimeter (ICESat\footnote{Something about ICESat.}) \parencite{Pritchard2012}, but the time span of this data set only covered the period 2003--2008 (5 years). In contrast, satellite radar altimeters (RA) have been providing
measurements of the ice shelves since 1978\footnote{Although another radar altimeter flew before
this date over the ice sheets (the GEOS-3), its experimental measurements were not
useful for ice-sheet change determination.} \parencite{Zwally1983, Zwally1989, Davis1998, Martin1983}. However,
historical RA missions like Seasat (1978) and Geosat (1985--1989) were limited to orbit latitudes north of 72\degree S (south of 72\degree N for Greenland) and, although insufficient for whole ice sheet studies, this coverage did capture a large portion of the fringing ice shelves and marginal grounded ice.

A satellite radar altimeter measures the satellite-to-surface round-trip travel time of the emitted electromagnetic waves\footnote{The standard altimeters used in this study operated in the Ku-band (microwaves): $f =$ 12--18 GHz, $\lambda =$ 2.5--1.7 cm} and their phase shift (Figure \ref{ch1figX}). As electromagnetic waves travel through the atmosphere, they can be delayed by water vapour or by ionization. Once these effects are corrected for, the final range $R$ is estimated with high precision\footnote{The strength of satellite radar altimeters is their single-measurement high precision, which is fundamental for estimating changes (more so than accuracy).}. The radar-altimeter functional response over ice surfaces is considerably more complex than over the oceans. Causal factors identified in the complex backscatter response over ice sheets include sloping surfaces, surface undulations with characteristic wavelengths on the same spatial scale as the altimeter beam-limited footprint, off-track reflections, dynamic lag of the altimeter tracking circuit (on-board, which requires an off-board post-processing \emph{retracking} step), and spatio-temporal variations in the ice-surface properties. \parencite{Martin1983}. \emph{Retracking} methods using the altimeter return pulse waveforms (Figure \ref{ch1figX}) give range corrections that are typically several meters, and significant differences are found between different \emph{retracking} algorithms \parencite{Davis1996}. We then have the ellipsoidal height estimated by an altimeter as

\begin{equation}
  h = h_\text{s} - R - \epsilon
\end{equation}

where $h$ is the height of the surface with respect to the ellipsoid, $h_\text{s}$ is the satellite's altitude, $R$ is the satellite-to-surface measured range, and $\epsilon$ represents the atmospheric path delay (accounted for). By performing repeated measurements one can track changes in the ice-shelf height, that relates to the ice-shelf mass balance as follows

\begin{equation}
\begin{split}
  \frac{\partial h}{\partial t} &= \frac{\partial \Delta}{\partial t}
    - M \, \frac{\partial}{\partial t} \, \rho^{-1}_\text{w}
    + \int^M_0 \text{d}m \, \frac{\partial}{\partial t} \, \rho^{-1}_\text{f}(m) \\
    &\quad + ( \rho^{-1}_\text{i} - \rho^{-1}_\text{w} ) 
              \left( \dot{M}_\text{s} + \dot{M}_\text{b} + \vect{u} \cdot \nabla M
              + M \nabla \cdot \vect{u} \right)
\end{split}
\end{equation}

where...[explanation of each component]

With the demonstrated capability of all-wheather-operational radar altimeters
(Figure: RA measurement scheme)
a new era of polar science was set. More than two decades of modern (global) RA missions,
including the ERS-1 (1992--1996), the ERS-2 (1995--2003), the Envisat (2002--2012), and
the Cryosat-2 (2010--present), have improved dramatically our understanding of ice-sheet
dynamics and its relation to measured sea-level rise [CITE?]. In general, however,
studies focusing on the ice shelves using RA have under-exploited these data,
reporting linear height-trends for a fraction---and usually for broad regions---of the complete (modern-era) RA data set \parencite{Shepherd2003, Shepherd2010, Zwally2005}.

In addition, satellite observations over the past two decades have provided estimates of
the Antarctic mass budget, which vary widely according the method they are based upon---from
$+$50 to $-$250 Gt year$^{-1}$ for 1992--2009 \parencite{Zwally2011}. As reported in
\textcite{Zwally2011}:

\begin{quotation}
\noindent
Generally, the range of estimates in IPCC07\footnote{Define IPCC07.} encompassed the errors listed in the studies, but as noted in the report, a mid-range value does not indicate a more reliable estimate, and the composite errors listed in each study do not define confidence limits because important components lack formal statistical derivation. This caution on the error estimates applies to all studies regardless of methods of data collection and analyses.
\end{quotation}

This highlights the need to account for uncertainties in a more consistent way, as well as analyzing these inherently noisy and complex measurements with robust statistical estimation procedures.

\section*{Scientific objectives}

[SEE "ORGANIZATION" EXAMPLE OF BOOK: TREES, MAPS, AND...]

This PhD thesis seeks to address the following scientific questions:

\begin{enumerate}
  \item[a)] Has $\partial h / \partial t$ for Antarctic ice shelves been constant
  or has it varied over time?
  \item[b)] Are the observed changes restricted to particular regions or do they
  occur in a larger scale?
  \item[c)] What are the spatial patterns of coherence in $\partial h / \partial t$
  around Antarctica?
  \item[d)] Are there links between interannual variability of the ice shelves and
  oceanic and atmospheric variability?
\end{enumerate}

To be achieved with the conclusion of the following objectives:

\begin{enumerate}
  \item[i)] {\it Derive reliable time series of elevation change} ({\sc Chapter 2}). Improve and
  extend the procedures for extracting height changes from multi-mission RA
  records, deriving reliable long-term, continuous time series in a
  high-resolution grid for consistent trend analysis
  \item[ii)] {\it Quantify long-term trends} ({\sc Chapter 3}). Analyze long-term changes for
  all Antarctic ice shelves spanning a time period of nearly two decades,
  mapping in time and space the height-/volume-change rates, acceleration
  and associated uncertainties.
  \item[iii)] {\it Analyze interannual-to-decadal variability} ({\sc Chapter 4}). Use the
  extended time series and spatial maps to seek relationships between
  ice-shelf-height fluctuations and known modes of variability in the
  atmosphere and the ocean ({\it e.g.}, El Ni\~{n}o Southern Oscillation).
\end{enumerate}

\section*{Thesis outline}

This thesis is organized in four chapters. Each chapter was designed as a self-contained piece of work to address the general objectives proposed in this dissertation (above). {\sc Chapter~1} (this current chapter) presents an introduction to the scientific questions being addressed and the background necessary to follow the work done in Chapters 2--4. {\sc Chapter~2} describes the full methodology implemented in this study (partially developed and partially modified from previous works) to construct the longest, continuous, high-resolution (in time and space) time series of changes in ice-shelf surface-height from multiple satellite radar altimeters. In this chapter the RA data and corrections are discussed, and new processing approaches are proposed. This chapter is currently in review for publication in the journal {\it Remote Sensing of Environment}. {\sc Chapter~3} describes the trend analysis approach implemented in the ice-shelf-height time-series product (derived in Chapter 2), as well as reports and discusses the findings, which have advanced significantly our understanding on the state of the Antarctic ice shelves. This chapter is published in the journal {\it Science}. {\sc Chapter~4} focuses on the variability analysis of the ice-shelf-height time series. This chapter investigates the origin of interannual fluctuations in the ice-shelf volume, and seeks to understand the links between observe changes and large-scale climate variability. This chapter is currently in preparation for submission \emph{after} the dissertation defense.

\section*{Summary of results}

This thesis presents a method for optimal analyses of ice-shelf-height data from multiple satellite RAs and estimation of uncertainties in the resulting data products. We constructed an 18-year time series of height changes at $\sim$30 km grid cells and $\sim$3 month intervals over Antarctica's floating ice shelves. Our data set allowed us to estimate, with high statistical confidence (95\% level), the temporal progression and spatial structure of ice-shelf height changes in Antarctica between 1994 and 2012.

We have demonstrated that: Substantial averaging both in time and space is required to construct RA height records over floating ice shelves. Densification of the surface greatly affects the height-change estimate, and backscatter correction significantly reduces this effect. Densification is a more important effect than penetration in biasing the height-change estimates over the ice shelves. Given the high interannual-to-decadal variability that is present, a simple straight-line fit fails to capture the underlying trends, some degree of curvature in the trend is needed. Polynomial trends allow to obtain information on the evolution and spatial structure of changes, such as instantaneous rate of change (derivative of the trend) and average acceleration (slope of the derivative). Given the convoluted nature of the error sources, we propose that a top-down approach for uncertainty estimation (e.g., bootstrap applied to time-dependent data) constitutes a more accurate alternative for error analysis.

We have shown that Antarctic ice-shelf volume
loss is accelerating. In the Amundsen Sea,
some ice shelves buttressing regions of grounded
ice that are prone to instability have experienced
sustained rapid thinning for almost two decades.
If the present climate forcing is sustained, we
expect a drastic reduction in volume of the rapidly
thinning ice shelves at decadal to century
time scales, resulting in grounding-line retreat
and potential ice-shelf collapse. Both of these processes
further accelerate the loss of buttressing,
with consequent increase of grounded-ice
discharge and sea-level rise. On smaller scales,
ice-shelf thickness variability is complex, demonstrating
that results from single satellite missions
with typical durations of a few years are
insufficient to draw conclusions about the long-term
response of ice shelves. Large changes occur
over a wide range of time scales, with rapid variations
of ice-shelf thickness suggesting that ice
shelves can respond quickly to changes in oceanic
and atmospheric conditions.

\section*{Future research}

Our preliminary results on the variability analysis performed to the 18-year record of ice-shelf height changes, show that there is significant interannual fluctuations in Antarctic ice-shelf volume. Besides advancing our understanding on how the oceanic and atmospheric forcing impacts the ice sheet, this has also important implications for estimating changes in freshwater fluxes around Antarctica. With the aid of innovative mathematical tools for analyzing sets of short and noisy records (e.g., multi-variate singular spectrum analysis), we have estimated the spectrum of sub-decadal changes for some key regions in West Antarctica. We found statistically significant peaks (energy content) in two particular frequency bands: $\sim$1 year (the expected contribution from the seasonal cycle) and $\sim$4--6 years. Interestingly, the latter coincides with the characteristic spectral signature of El Ni\~no Southern Oscillation (ENSO) as derived from the Sourthern Oscillation Index (SOI). The ENSO has been shown to affect sea-surface pressure and, consequently, sea-ice conditions in the Amundsen and Bellingshausen seas. Although the Antarctic response to ENSO forcing is not ubiquitous, it is reasonable to expect that this large-scale climate variation will be felt by the floating ice shelves as well. Investigating potential linkages between tropical-climate variability and changes in the Antarctic ice shelves is the subject of a planned postdoctoral project.
