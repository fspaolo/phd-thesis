%
% Front pages
% -----------
% Jun 1, 2015 
%
%


%% REQUIRED FIELDS -- Replace with the values appropriate to you

% No symbols, formulas, superscripts, or Greek letters are allowed
% in your title.
\title{
Interannual and decadal variations of Antarctic ice shelves using
multi-mission satellite radar altimetry, and links with oceanic and atmospheric
forcings
}

\author{Fernando Serrano Paolo}
\degreeyear{2015}

% Master's Degree theses will NOT be formatted properly with this file.
\degreetitle{Doctor of Philosophy} 

\field{Earth Sciences}

\chair{Helen A. Fricker}
% Uncomment the next line iff you have a Co-Chair
% \cochair{Professor Cochair Semimaster} 
%
% Or, uncomment the next line iff you have two equal Co-Chairs.
%\cochairs{Professor Chair Masterish}{Professor Chair Masterish}

%  The rest of the committee members  must be alphabetized by last name.
\othermembers{
Sarah T. Gille\\
Falko Kuester\\
Jean-Bernard Minster\\
Laurie Padman\\ 
David T. Sandwell\\
}
\numberofmembers{6} % |chair| + |cochair| + |othermembers|


%% START THE FRONTMATTER
%
\begin{frontmatter}

%% TITLE PAGES
%
%  This command generates the title, copyright, and signature pages.
%
\makefrontmatter 

%% DEDICATION
%
%  You have three choices here:
%    1. Use the ``dedication'' environment. 
%       Put in the text you want, and everything will be formated for 
%       you. You'll get a perfectly respectable dedication page.
%   
%    2. Use the ``mydedication'' environment.  If you don't like the
%       formatting of option 1, use this environment and format things
%       however you wish.
%
%    3. If you don't want a dedication, it's not required.
%
%
%\begin{dedication} 
%  To someone...
%\end{dedication}


\begin{mydedication} % You are responsible for formatting here.
  \vfil
  \centering
  \emph{
    To my grandma Nira,\\
    my grandpa Horacio,\\
    and my beautiful Charlotte.
  }
  \vfil
\end{mydedication}
\clearpage


%% EPIGRAPH
%
%  The same choices that applied to the dedication apply here.
%
%\begin{epigraph} % The style file will position the text for you.
%\emph{
%It should never assert as argument the authority of any man, by excellent and illustrious that is. It is extremely unfair to fold one's own feeling to a slavish reverence towards others; being submissive is worthy of mercenaries or slaves and contrary to the dignity of human freedom. It is utmost stupidity to believe by inveterate custom. It is irrational thing to conform with an opinion because of the number of those who have it. We must instead always seek a real and necessary reason... \\
%and listen to the voice of nature.
%} \\
%---Giordano Bruno \\
%(burned at the stake, 1548--1600)
%\end{epigraph}

\begin{myepigraph} % You position the text yourself.
  \vfil
  \centering
  \begin{minipage}{.75\textwidth}
    {\justify
    \emph{
    It should never stand as argument the authority of any man, regardless of how excellent and illustrious he is... it is grossly unfair to fold one's own feeling to a submissive reverence toward another; it is worthy of mercenaries or slaves and contrary to the dignity of human freedom to suppress oneself and to be submissive; it is supreme stupidity to believe by inveterate custom; it is an irrational thing to conform with an opinion because of the number of those who have it... it has to be sought, instead, always a reason, true and necessary... and listen to the voice of nature.
    }
    }
    \begin{flushright}
    ---Giordano Bruno 1548--1600 (burned at the stake)
    \end{flushright}
  \end{minipage}
  \vfil
\end{myepigraph}


%% SETUP THE TABLE OF CONTENTS
%
\setcounter{tocdepth}{1}
\tableofcontents
\listoffigures  % Uncomment if you have any figures
\listoftables   % Uncomment if you have any tables



%% ACKNOWLEDGEMENTS
%
%  While technically optional, you probably have someone to thank.
%  Also, a paragraph acknowledging all coauthors and publishers (if
%  you have any) is required in the acknowledgements page and as the
%  last paragraph of text at the end of each respective chapter. See
%  the OGS Formatting Manual for more information.
%
\begin{acknowledgements} 

I would like to acknowledge all the people who have assisted me throughout my
dissertation research. In particular, my advisor Helen Fricker and co-advisor Laurie Padman, for their support and guidance throughout my entire PhD work. They were key for the success of my doctorate.

Additionally, I would like to express my sincere gratitude for the service of my
committee members, David Sandwell, Jean-Bernard Minster, Sarah Gille and Falko Kuester, in addition to Helen and Laurie.

Finally, I would like to thank all the denizens of IGPP for being so supportive of
me during my time at SIO. Many thanks to my good friends, Valerie Sahakian, Robert Petersen, Diego Melgar, Samer Naif, Soli Garcia, Katia Tymofyeyeva, Rachel Marcuson and Ashlee Henig. A very special thank you to Gedilani Oliveira and Catia Perciani; and of course, to my lovely Charlotte.

{\sl Chapter 2}, in full, is in revision for publication of the material as it
may appear in {\it Remote Sensing of Environment} 2015. Paolo, Fernando S.;
Fricker, Helen A.; Padman, Laurie. The dissertation author was the primary
investigator and author of this paper.

{\sl Chapter 3}, in full, is a reprint of the material as it appears in {\it Science}
2015. Paolo, Fernando S.; Fricker, Helen A.; Padman, Laurie. The dissertation
author was the primary investigator and author of this paper.

{\sl Chapter 4}, in full, is currently being prepared for submission for publication
of the material. Paolo, Fernando S.; Fricker, Helen A.; Padman, Laurie. The
dissertation author was the primary investigator and author of this material.
 
\end{acknowledgements}


%% VITA
%
%  A brief vita is required in a doctoral thesis. See the OGS
%  Formatting Manual for more information.
%
\begin{vitapage}

\begin{vita}
  \item[2007] B.S. in Oceanography, University of S\~ao Paulo, Brazil
  \item[2009] M.S. in Geophysics, University of S\~ao Paulo, Brazil
  \item[2015] Ph.D. in Earth Sciences, University of California, San Diego\\
              Curricular group: Geophysics
\end{vita}

\begin{awards}
  \item[2014] AGU Outstanding Student Paper Award, Cryosphere
  \item[2013--15] NASA Earth and Space Science Fellowship (NESSF)
  \item[2011] AAAS Student Award (1st place), Atmospheric \& Oceanographic Sci.
  \item[2010] AGU Outstanding Student Paper Award, Cryosphere
  \item[2010] Honorable Mention (best M.S. Thesis in Geophysics), Univ. of S\~ao Paulo
  %2008 & T.A. Fellowship (M.S.), Brazilian Ministry of Education
  \item[2007--08] Brazilian Ministry of Sci. \& Tech. Fellowship (Masters)
  \item[2007] Honorable Mention (2nd best B.S. Thesis), Univ. of S\~ao Paulo
  \item[2005--06] S\~ao Paulo Research Foundation Fellowship (Undergrad)
  \item[2004] Brazilian Ministry of Sci. \& Tech. Fellowship (Undergrad)
\end{awards}

\begin{publications}
  \item {\bf F.~S.~Paolo}, H.~A.~Fricker, L.~Padman, "Developing improved
        decadal records of Antarctic ice-shelf height change from multiple
        satellite radar altimeters", \emph{Remote Sens. Environ.} (2015) in revision.
  \item P.~R.~ Holland, A.~Brisbourne, H.~F.~J.~Corr, D.~McGrath, K.~Purdon, 
        J.~Paden, H.~A.~Fricker, {\bf F.~S.~Paolo}, A.~Fleming, "Oceanic and 
        atmospheric forcing of Larsen C Ice-Shelf thinning", \emph{The Cryosphere} 
        (2015).
  \item {\bf F.~S.~Paolo}, H.~A.~Fricker, L.~Padman, "Volume loss 
        from Antarctic ice shelves is accelerating", \emph{Science} (2015).
  \item {\bf F.~S.~Paolo}, E.~C.~Molina, "Integrated marine 
        gravity field along the Brazilian coast from altimeter-derived sea 
        surface gradient and shipborne gravity", \emph{J. Geodyn.} (2010).
  \item {\bf F.~S.~Paolo}, "Satellite altimetry and marine gravity on the
        integrated representation of the gravity field along the Brazil coast",
        {\it M.S. Thesis} (2009).
  \item M.~C.~B\'icego, E.~Zanardi-Lamardo, S.~Taniguchi, C.~C.~Martins, 
        D.~A.~M.~da~Silva, S.~T.~Sasaki, A.~C.~R.~Albergaria-Barbosa, {\bf F.~S.~Paolo},
        R.~R.~Weber, R.~C.~Montone, "Results from a 15-year 
        study on hydrocarbon concentrations in water and sediment from 
        Admiralty Bay, King George Island, Antarctica", \emph{Antarct. Sci.} 
        (2009).
  \item {\bf F.~S.~Paolo}, M.~M.~Mahiques, "Utilization of 
        acoustic methods in coastal dynamics studies: example in the 
        Canan\'eia lagoonal mouth", \emph{Braz. J. Geophys.} (2008).
\end{publications}

\begin{coursework}
  \item Studies in {\sl Ice and the Climate System} --- Prof. H. A. Fricker
  \item Studies in {\sl Physics of Global Warming} --- Prof. R. Keeling
  \item Studies in {\sl Physics of Earth Materials (Continuum Mechanics)} --- Prof. D. C. Agnew
  \item Studies in {\sl Internal Constitution of the Earth} --- Profs. G. T. Masters \& D. Stegman
  \item Studies in {\sl Applied Mathematics} --- Prof. E. Lauga 
  \item Studies in {\sl Inverse Theory} --- Prof. C. Constable 
  \item Studies in {\sl Geophysical Data Analysis} --- Profs. R. L. Parker \& D. C. Agnew
  \item Studies in {\sl Oceanographic Data Analysis} --- Prof. D. Rudnick 
  \item Studies in {\sl Computer Intensive Statistics} --- Dr. J. Barlow 
  \item Studies in {\sl Numerical Methods for Partial Differential Eqs.} --- Prof. Y. Fialko
  \item Studies in {\sl Seismology} --- Profs. G. T. Masters \& J-B. Minster
  \item Studies in {\sl Geodynamics} --- Prof. D. T. Sandwell
  \item Studies in {\sl Gravity and Geomagnetism} --- Profs. R. L. Parker \& C. Constable
  \item Studies in {\sl Satellite Remote Sensing} --- Prof. D. T. Sandwell \& H. A. Fricker
  \item Studies in {\sl Ethical \& Professional Science} --- Profs. S. Constable \& C. Constable
\end{coursework}

\end{vitapage}


%% ABSTRACT
%
%  Doctoral dissertation abstracts should not exceed 350 words. 
%   The abstract may continue to a second page if necessary.
%
\begin{abstract}
Antarctica's ice shelves, the floating extensions of the ice sheet, exert an important dynamic constraint on the flow of ice from the grounded ice sheet to the ocean, and hence on changes in global sea level. Thinning of an ice shelf reduces its ability to restrain the ice discharge from the grounded ice-sheet interior. Since the grounded ice sheet responds to perturbations in the ice shelves, predicting sea-level rise requires that we understand the processes that determine ice-shelf response to climate variability. Our understanding of these processes is, however, still too rudimentary to allow prediction of ice-sheet change under projected future climate states. This dissertation presents improved procedures to construct 18-year time series (1994-2012) of ice-shelf height around the entire Antarctic continent by merging data from multiple overlapping satellite radar altimeter missions (ERS-1, ERS-2, and Envisat). The resulting data set has a temporal resolution of 3 months and a spatial resolution of $\sim$30 km. Improved procedures for trend analysis are introduced, and a more accurate alternative for uncertainty estimation to the standard error propagation approach is proposed. Ice-shelf height variability is analyzed using orthogonal-component decomposition of multivariate time series, spectral estimation and background-noise statistical tests. The derived data set and method allow to estimate, reliably and with defined formal uncertainties, the temporal progression and spatial structure of changes in ice-shelf height and volume in Antarctica between 1994 and 2012. The results reveal that, overall, Antarctic ice-shelf volume loss is accelerating. Furthermore, significant interannual variability in the Amundsen Sea ice shelves is strongly correlated with the low-frequency mode of El Ni\~no-Southern Oscillation. These findings may ultimately allow us to understand the processes driving ice-shelf changes sufficiently to improve our models for predicting future ice loss.
\end{abstract}


\end{frontmatter}
